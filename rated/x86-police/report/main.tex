%%%%%%%%%%%%%%%%%%%%%%%%%%%%%%%%%%%%%%%%%%%%%%%%%%%%%%%%%%%%%%%%%%%%%%%%%%%%%%%%%
%                                DO NOT EDIT                                    %
%%%%%%%%%%%%%%%%%%%%%%%%%%%%%%%%%%%%%%%%%%%%%%%%%%%%%%%%%%%%%%%%%%%%%%%%%%%%%%%%%
\documentclass[a4paper, 11pt]{article}
\usepackage[top=2.5cm, bottom=2.5cm, left=2cm, right=2cm]{geometry}
\geometry{a4paper}
\usepackage[utf8]{inputenc}
\usepackage{textcomp}
\usepackage{graphicx}
\usepackage{amsmath,amssymb}
\usepackage{bm}
\usepackage[pdftex,bookmarks,colorlinks,breaklinks]{hyperref}
\usepackage{memhfixc}
\usepackage{pdfsync}
\usepackage{fancyhdr}
\usepackage{listings}
\usepackage{xcolor}

\definecolor{codegreen}{rgb}{0,0.6,0}
\definecolor{codegray}{rgb}{0.5,0.5,0.5}
\definecolor{codepurple}{rgb}{0.58,0,0.82}
\definecolor{backcolour}{rgb}{0.95,0.95,0.92}

\lstdefinestyle{codestyle}{
    backgroundcolor=\color{backcolour},
    commentstyle=\color{codegreen},
    keywordstyle=\color{magenta},
    numberstyle=\tiny\color{codegray},
    stringstyle=\color{codepurple},
    basicstyle=\ttfamily\footnotesize,
    breakatwhitespace=false,
    breaklines=true,
    captionpos=b,
    keepspaces=true,
    numbers=left,
    numbersep=5pt,
    showspaces=false,
    showstringspaces=false,
    showtabs=false,
    tabsize=2
}
\lstset{style=codestyle}
\pagestyle{fancy}
\pagenumbering{gobble}
%%%%%%%%%%%%%%%%%%%%%%%%%%%%%%%%%%%%%%%%%%%%%%%%%%%%%%%%%%%%%%%%%%%%%%%%%%%%%%%%%

% DO NOT EDIT the structure of the document and stick to the provided template!
% The report must be in English and PDF format. Do not send files other than the report.
% It must be at most 1 page long, not counting images, code and references.
% Images and code should be included in the report only if they add useful information.
% A person reading the report should be able to understand the vulnerability and exploit it without looking for further information.

\title{x86-linux-police}
\author{Filippo De Grandi}
\date{13/04/2025}

\makeatletter
\lhead{\@author,\space\@date}
\rhead{\@title}
\lfoot{Ethical Hacking 2024/25}
\rfoot{University of Trento}

\begin{document}

\section*{Background}
% Briefly introduce the background on the challenge topic. Describe the class of the vulnerability that affects the target system, why it happens, and its consequences. You might also describe how the vulnerability can be prevented. In this section, you should not describe the challenge details.
% The background section should be well referenced \cite{bibtex}.
The challenge is a reverse engineering exercise that focuses on retrieving the flag from an ELF binary file \cite{elf}.

Binary reverse engineering \cite{rev_eng} is the process of analyzing compiled software to understand how it functions without access to its source code.
By examining the binary code, analysts can deduce the underlying algorithms, identify vulnerabilities, and repurpose or modify the software. This technique is essential for security assessments, malware analysis, and gathering insights into proprietary systems.

While complete prevention of reverse engineering is impossible, several techniques can complicate the process for attackers. One common approach is stripping, which removes symbols such as function names and global variable names, obscuring the code's intent \cite{stripping}.

Another method involves anti-disassembly techniques, where specific bytes are inserted to exploit limitations in disassemblers (e.g., Ghidra). This manipulation can mislead analysis tools without affecting the program's execution \cite{anti_disass}.

Additional methods include code packing, which compresses or encrypts the code to make analysis more difficult \cite{packing}, and virtual machine obfuscation, where the code is transformed into a format that runs on a custom virtual machine \cite{vm_obf}.


\section*{Reversing}

The binary uses several techniques to make reverse engineering more difficult:

\paragraph{Anti-Disassembly}\label{par:Anti-Disassembly} % (fold)
The UD2 instructions trigger SIGILL exceptions, causing disassemblers such as Ghidra to misinterpret the following bytes as data.
% paragraph Anti-Disassembly (end)


\paragraph{Signal Handling Obfuscation}\label{par:Signal Handling Obfuscation} % (fold)
A custom SIGILL handler, established using a sigaction structure, redirects execution to the flag-checking routine. This method hides the actual control flow, making it harder to determine which address holds the user input and which the flag itself.
% paragraph Signal Handling Obfuscation (end)

\paragraph{Stripped Symbols}\label{par:Stripped Symbols} % (fold)
The removal of function names and variable identifiers requires more in-depth manual analysis, as no symbol information is provided to indicate the binary's structure.
% paragraph Stripped Symbols (end)

\section*{Solution}
% Describe the solution of the challenge and the exploit.
% You can include snippets of code, such as in Listing \ref{lst:example}
The program when run outputs \texttt{"Gimme the flag to check:"}. After disassembling the binary with Ghidra, I searched for that string in but could not find it.
Since it is a stripped binary, I could not directly jump to \texttt{main}, but finding it was possible via the \texttt{entry} function that calls \texttt{\_\_libc\_start\_main}, containing the address of \texttt{main}.

In the \texttt{main} function, an \texttt{invalidInstructionException} function is called, that is obfuscated via UD2 instruction. Ghidra did not disassemble the code, but it can be forced to do so.
It is in this disassembled function that the program reads the user input, and I found out that the user input is saved inside \texttt{DAT\_00104060}.

Back in the \texttt{main} function, the function pointer at \texttt{FUN\_00101179} is assigned to a \texttt{sigaction handler}. In that function, each byte of the user input XORED with \texttt{0x42} is compared with a stored variable, presumably the flag.
By copying the bytes inside the variable with \textit{Copy Special $\rightarrow$ Python Byte String}, the contents of the flag can be retrieved. Listing \ref{lst:sol} outputs the flag.

\newpage


\lstinputlisting[language=python,label=lst:solution,caption=Solution code, label=lst:sol]{solution.py}

\bibliographystyle{abbrv}
\bibliography{references}  % need to put bibtex references in references.bib
\end{document}
