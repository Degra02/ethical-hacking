%%%%%%%%%%%%%%%%%%%%%%%%%%%%%%%%%%%%%%%%%%%%%%%%%%%%%%%%%%%%%%%%%%%%%%%%%%%%%%%%%
%                                DO NOT EDIT                                    %
%%%%%%%%%%%%%%%%%%%%%%%%%%%%%%%%%%%%%%%%%%%%%%%%%%%%%%%%%%%%%%%%%%%%%%%%%%%%%%%%%
\documentclass[a4paper, 11pt]{article}
\usepackage[top=2.5cm, bottom=2.5cm, left=2cm, right=2cm]{geometry}
\geometry{a4paper}
\usepackage[utf8]{inputenc}
\usepackage{textcomp}
\usepackage{graphicx}
\usepackage{amsmath,amssymb}
\usepackage{bm}
\usepackage[pdftex,bookmarks,colorlinks,breaklinks]{hyperref}
\usepackage{memhfixc}
\usepackage{pdfsync}
\usepackage{fancyhdr}
\usepackage{listings}
\usepackage{xcolor}

\definecolor{codegreen}{rgb}{0,0.6,0}
\definecolor{codegray}{rgb}{0.5,0.5,0.5}
\definecolor{codepurple}{rgb}{0.58,0,0.82}
\definecolor{backcolour}{rgb}{0.95,0.95,0.92}

\lstdefinestyle{codestyle}{
    backgroundcolor=\color{backcolour},
    commentstyle=\color{codegreen},
    keywordstyle=\color{magenta},
    numberstyle=\tiny\color{codegray},
    stringstyle=\color{codepurple},
    basicstyle=\ttfamily\footnotesize,
    breakatwhitespace=false,
    breaklines=true,
    captionpos=b,
    keepspaces=true,
    numbers=left,
    numbersep=5pt,
    showspaces=false,
    showstringspaces=false,
    showtabs=false,
    tabsize=2
}
\lstset{style=codestyle}
\pagestyle{fancy}
\pagenumbering{gobble}
%%%%%%%%%%%%%%%%%%%%%%%%%%%%%%%%%%%%%%%%%%%%%%%%%%%%%%%%%%%%%%%%%%%%%%%%%%%%%%%%%

% DO NOT EDIT the structure of the document and stick to the provided template!
% The report must be in English and PDF format. Do not send files other than the report.
% It must be at most 1 page long, not counting images, code and references.
% Images and code should be included in the report only if they add useful information.
% A person reading the report should be able to understand the vulnerability and exploit it without looking for further information.

\title{Bizarre Auction}
\author{Filippo De Grandi}
\date{25/03/2025}

\makeatletter
\lhead{\@author,\space\@date}
\rhead{\@title}
\lfoot{Ethical Hacking 2024/25}
\rfoot{University of Trento}

\begin{document}

\section*{Background}
% Briefly introduce the background on the challenge topic. Describe the class of the vulnerability that affects the target system, why it happens, and its consequences. You might also describe how the vulnerability can be prevented. In this section, you should not describe the challenge details.
% The background section should be well referenced \cite{bibtex}.
The challenge is about a web application that allows users to bid on a selection of items.
The system is vulnerable to SQL injection \cite{sqli} attacks, more precisely to the Blind Time-based \footnote{The challenge is vulnerable to Boolean-based SQLi as well, but a time-based injection has been used for the solution} SQL Injection technique \cite{blind-sqli}, which allows an attacker to infer the content of the database by observing the application's behavior without directly retrieving the data. \\
SQL injections are a type of attack in which SQL code is inserted into web forms, query strings, or HTTP headers that will be executed by the underlying DBMS.
Such an attack occurs when tainted data is used to construct SQL queries without proper validation or sanitization. \\
Such tainted data derives from user input, which can be manipulated by an attacker to execute arbitrary SQL code \cite{sqli-cwe}. \\
This type of attack has severe consequences, such as:
\begin{itemize}
    \item Data leakage: the attacker can retrieve sensitive information from the database, with consequent loss of Confidentiality.
    \item Data manipulation: the attacker can modify the content of the database, with loss of Integrity.
    \item Authentication bypass: the attacker can retrieve user credentials.
    \item Authorization bypass: the attacker can retrieve information that should not be accessible.
    \item Denial of Service: the attacker can execute queries that consume all the resources of the database.
\end{itemize}
To prevent SQL injections, it is essential to use parameterized queries (prepared statements), which separate the SQL code from the user input, thus preventing the injection of malicious code.
Allow-listing input validation can also be used to prevent the injection of unwanted characters and / or words \cite{sqli-prevention}.


\section*{Vulnerability}
% Briefly describe the specific instance of the vulnerability that affects the target system.
The vulnerability is found in the \texttt{offer} parameter of the \texttt{POST} request to the \texttt{\$URL/product/<prod-num>} endpoint, which would be used to bid on a specific item.
If a simple \texttt{sleep(2)} function is inserted instead of the offer, the application delays the response by a certain amount of time, confirming the presence of a SQL injection vulnerability.
The injection point is the same for both challenges, but the server-side validation is different.

\section*{Solution}
% Describe the solution of the challenge and the exploit.
% You can include snippets of code, such as in Listing \ref{lst:example}
In order to find the password of the admin user, first some table names have to be dumped. This can be done by sending a payload like the following: \\
\texttt{1 and (select sleep(<treshold>) from information\_schema.tables where table\_schema = DATABASE() And HEX(table\_name) like \'<name>\%\')} \\
and the name of the table can be built through the responses of the server at each query. \\
In the first challenge, the server checks for all occurrences of SELECT, FROM, WHERE, AND and OR (even in lowercase). This check can be bypassed by just capitalizing only the first letter of each keyword (even inside other words like \texttt{passwOrd}). This will be called \texttt{$\varepsilon_1$}. \\
In the second challenge, this check is improved by lowering the case of the whole query before the filtering. A hint of the underlying code, though, shows that the server is firstly performing this check, then normalizing the query into \texttt{NFKD} form \cite{unicode} since MySQL is not unicode-friendly. \\
To bypass the second version's sanitization, then, the solution is to replace the characters with alternative unicode versions of them that will then be normalized back into the their \textit{normal} form.
This will be called \texttt{$\varepsilon_2$} \footnote{No example is provided since LaTeX does not support these characters. Run the code to see some of them.}. \\

After dumping the \texttt{user} table name, the password can be obtained by sending a payload like the following: \texttt{1 and (select sleep(<treshold>) from user where username = 'admin' and hex(password) like \'<password>\%\'}, properly encoded either through \texttt{$\varepsilon_1$} or \texttt{$\varepsilon_2$} based on the version. \\

The solution code is provided in Listing \ref{lst:solution}.

\lstinputlisting[language=python,label=lst:solution,caption=Solution code]{solution.py}

\bibliographystyle{abbrv}
\bibliography{references}  % need to put bibtex references in references.bib
\end{document}
