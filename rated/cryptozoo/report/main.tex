%%%%%%%%%%%%%%%%%%%%%%%%%%%%%%%%%%%%%%%%%%%%%%%%%%%%%%%%%%%%%%%%%%%%%%%%%%%%%%%%%
%                                DO NOT EDIT                                    %
%%%%%%%%%%%%%%%%%%%%%%%%%%%%%%%%%%%%%%%%%%%%%%%%%%%%%%%%%%%%%%%%%%%%%%%%%%%%%%%%%
\documentclass[a4paper, 11pt]{article}
\usepackage[top=2.5cm, bottom=2.5cm, left=2cm, right=2cm]{geometry}
\geometry{a4paper}
\usepackage[utf8]{inputenc}
\usepackage{textcomp}
\usepackage{graphicx}
\usepackage{amsmath,amssymb}
\usepackage{bm}
\usepackage[pdftex,bookmarks,colorlinks,breaklinks]{hyperref}
\usepackage{memhfixc}
\usepackage{pdfsync}
\usepackage{fancyhdr}
\usepackage{listings}
\usepackage{xcolor}

\definecolor{codegreen}{rgb}{0,0.6,0}
\definecolor{codegray}{rgb}{0.5,0.5,0.5}
\definecolor{codepurple}{rgb}{0.58,0,0.82}
\definecolor{backcolour}{rgb}{0.95,0.95,0.92}

\lstdefinestyle{codestyle}{
    backgroundcolor=\color{backcolour},
    commentstyle=\color{codegreen},
    keywordstyle=\color{magenta},
    numberstyle=\tiny\color{codegray},
    stringstyle=\color{codepurple},
    basicstyle=\ttfamily\footnotesize,
    breakatwhitespace=false,
    breaklines=true,
    captionpos=b,
    keepspaces=true,
    numbers=left,
    numbersep=5pt,
    showspaces=false,
    showstringspaces=false,
    showtabs=false,
    tabsize=2
}
\lstset{style=codestyle}
\pagestyle{fancy}
\pagenumbering{gobble}
%%%%%%%%%%%%%%%%%%%%%%%%%%%%%%%%%%%%%%%%%%%%%%%%%%%%%%%%%%%%%%%%%%%%%%%%%%%%%%%%%

% DO NOT EDIT the structure of the document and stick to the provided template!
% The report must be in English and PDF format. Do not send files other than the report.
% It must be at most 1 page long, not counting images, code and references.
% Images and code should be included in the report only if they add useful information.
% A person reading the report should be able to understand the vulnerability and exploit it without looking for further information.

\title{Cryptic Cryptozoo}
\author{Filippo De Grandi}
\date{25/03/2025}

\makeatletter
\lhead{\@author,\space\@date}
\rhead{\@title}
\lfoot{Ethical Hacking 2024/25}
\rfoot{University of Trento}

\begin{document}

\section*{Background}
% Briefly introduce the background on the challenge topic. Describe the class of the vulnerability that affects the target system, why it happens, and its consequences. You might also describe how the vulnerability can be prevented. In this section, you should not describe the challenge details.
% The background section should be well referenced \cite{bibtex}.

The challenge is a cryptographic exercise that centers on exploiting the vulnerabilities of AES-CBC encryption \cite{aes} \cite{cbc}.
One inherent characteristic of CBC mode is its malleability, which means that an attacker can manipulate the ciphertext by performing bit-flipping attacks \cite{bit-flipping} to induce controlled changes in the decrypted plaintext. This is possible because alterations to one ciphertext block affect the decryption of the subsequent block, allowing an adversary to modify specific parts of the plaintext without knowing the encryption key. \\
The vulnerability arises because AES-CBC does not provide any message integrity checks. Without incorporating additional measures, such as a Message Authentication Code (MAC) \cite{mac} or employing authenticated encryption modes like AES-GCM \cite{gcm}, the system remains vulnerable to modification attacks even when the encryption algorithm itself is considered secure. The lack of integrity verification permits unauthorized alterations that can result in significant security issues. \\
These vulnerabilities can have severe consequences, including unauthorized data manipulation, loss of message integrity, and exposure of sensitive information. The absence of proper integrity checks can allow attackers to alter configuration data, tamper with transaction details, or inject malicious commands. \\
Countermeasures include using authenticated encryption schemes or robust integrity checks, essential to mitigate the mentioned risks and ensure confidentiality and integrity of the data.


\section*{Vulnerability}
% Briefly describe the specific instance of the vulnerability that affects the target system.
In the challenge, a string representing a zoo gets created. The format of the plaintext, where fixed tokens (i.e., "pet=", delimiters, and padding elements) are concatenated with user-supplied data, creates a predictable layout. \\
This string of animal names is then encrypted using AES in CBC mode with a fixed Initialization Vector, no integrity checks and a well‐defined (and known) plaintext structure, enabling for the retrieval of the flag. \\
The injection point is in the same place for both versions, but the string concatenation and sanitization differs one from the other.


\section*{Solution}
% Describe the solution of the challenge and the exploit.
% You can include snippets of code, such as in Listing \ref{lst:example}
The program will return the flag if in the zoo there are some specific animals, but it performs sanitization before creating the zoo string. The block size for CBC is 16 bytes for both challenges.
In the first version, it needs to contain both \textit{Ferris} and \textit{rubberduck}. Since sanitization checks if either of the animals is inserted by the user, the solution is to insert some similar names and perform a bit-flip attack to change some characters during decryption. A possible block configuration can be the one found in \texttt{first\_exploit} function in Listing \ref{lst:sol}. \\
CBC decryption works as follow: $ P_i = D_K(C_i) \oplus C_{i-1} $, meaning that if a portion of the previous chipertext block $C_{i-1}$ gets purposefully changed, it would cause the output of $D_K(C_i)$ to get XORed with the tampered portion. \\
In the first version, by flipping some bits of the IV, the first block \texttt{pet=Ferriy|pet=u} can be changed to obtain \texttt{pet=Ferris|pet=r} which solves the challenge. \\
The second version is a bit more cumbersome as on top of requiring \textit{BillTheRock} to also be in the zoo to obtain the flag, it limits the number of bytes that can differ in the ciphertext to less than 5 (as the ciphertext gets hex-encoded), restricting the possible tampering. \\
This can be bypassed by flipping only either the low or high nibble of some ciphertext characters, so only half of the hex-version changes. The animal names have need to contain specific characters that differ from the ones that need to be obtained only by a nibble. \\
A possible block configuration can be the one in \texttt{second\_exploit} function in Listing \ref{lst:sol}. \\


\lstinputlisting[language=python,label=lst:solution,caption=Solution code, label=lst:sol]{solution.py}

\bibliographystyle{abbrv}
\bibliography{references}  % need to put bibtex references in references.bib
\end{document}
